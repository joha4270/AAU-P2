\todo{Add a title}
I alt vores musikafspillning indtil videre har vi brugt biblioteket midi-dot-net\cite{midi-dot-net} til at afspille musik i MIDI formatet over windows MIDI synthisiser\footnote{Vi har også eksperimenteret med andre synthisisers som føre til bedre lydkvalitet men det er de samme toner der bliver afspillet.}.\\
Vores første forsøg på afspilning af lyd starter det simpleste vi kunne finde. Vi afspillede tilfældige toner i rækkefølge. Resultaterne er lige så dårlige som man skulle tro, men det afspillede lyd og var derfor en vigtig milesten.\\
Andet forsøg var en overbygning på vores første musikgenerator\todo{right word?}. Den afspillede stadig tilfældige toner, men den kunne kun springe en enkelt tone af gangen, hvilket betød at der ikke længere var mærkelige spring i tonen men generatoren havde stadig ingen forståelse for rytme, harmoni eller melodi.\\
Vores tredje forsøg var det første forsøg skrevet med faktisk brug af musikteori. En liste af hvilke toner der lyder harmonisk sammen er blevet lavet i forvejen og generatoren vælger tilældigt at tilføje en anden tone der lyder godt sammen med den første valgte. Tonernes hastighed er også for første gang dynamisk og den afpiller toner med en længde på en halv, hel, halvanden, dobbelt og firdobbelte takt samtidig med at den ikke afspiller en tone hen over en 4 takts grænse.\newline{}
Langt det meste af tiden generere den ikke impornerende musik men den har korte perioder der lyder godt. Det forventes at en generator som også indarbejder en faktisk forståelse for rytme og melodi vil øge ratio af hit til shit.