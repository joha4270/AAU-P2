Genetiske Algoritmer:
En genetisk Algoritme er en rekursions algoritme der efterligner evolutionsteori. En genetisk algoritme skal bruge en række krav, og en række nuværende objekter.
Når den genetiske algoritme kører prøver den at kombinere værdier fra de nuværende objekter for at leve op til de krav der bliver stillet. Hvis en af kombinationerne mellem objekterne lever op til kravene afsluttes algoritmen, ellers kalder algoritmen sig selv med resultatet af kombinationerne som de nye objekter.
En stor del af princippet bag genetiske algoritmer er de forskellige objekters tilpasning, hvilket er bestemt ud fra hvilke værdier hvert objekt består af. I starten af den genetiske algoritme fjernes de objekter med den laveste tilpasningsværdi.

En ulempe ved genetiske algoritmer er at algoritmen er afhænging af en slutværdi dvs. at algoritmen kører indtil den opnår en værdi som allerede er givet den, hvilket betyder at den ikke laver ny musik men nærmere prøver at efterligne allerede eksisterende musik. 
