For bedst at opfylde projektets læringsmål arbejdes i det objektorienterede programmeringssprog C\#.
Sproget har et indbygget bibliotek, \emph{System.Media}, til at håndtere bølgebaserede lydfiler. Biblioteket kan afspille musik fra en fil, samt skrive lydbølger til en wavefile. Denne repræsentation af lyd er fortræffelig til at lagre i forvejen optagede lydklip, men det at generere musik fra grunden bliver enten ekstremt teknisk eller meget begrænset.\todo{Citation Needed} \cite{CitationNeeded}

Skulle musik genereres med kode, er det mere oplagt blot at arbejde i toner og instrumenter, hvilket understøttes af MIDI-standarden.  Fordelen ved MIDI-formatet er at det er meget simpelt og filerne fylder generelt mindre end bølge-baserede lydformater. MIDI formatet er dog ikke lige så fleksibelt som de bølgebaserede formater, bl.a. da det ikke er baseret på bølger og at man derfor ikke kan lave lyrik i MIDI filer. Til denne rapports formål er det dog ikke et problem, da lyrik ligger uden for rapportens område, men blot kan lægges over i en evt. senere gennemarbejdning af musikken, skulle nogen ønske det.
\todo{Skriv: "Vi vælger at bruge MIDI"}