I verden findes der mange muligheder for at udleve sine musikalske drømme, i både stor og lille grad.
Det gælder lige fra erfarne musikere der har lavet musik det meste af deres liv, til børn eller andre interesserede der lige er kommet ind i branchen.
Derudover lytter næsten alle mennesker til musik i højere eller mindre grad.
Af netop denne grund er musikområdet et utroligt velbeskrevet, men meget kulturbetonet og mangefacetteret område.


Musik har mange former og kan tage mange retninger.
Det kunne være at den enkelte person kunne lide Jazz, rock eller dansk pop.
Dertil er det også muligt at individet ikke interesserer sig for musikteori.
Herved kan det være at et lille program, som Mario Paint Composer kan tilfredsstille individets musikalske behov.
På samme måde findes der mange kreative måder man kan programmere musik på, og det er dette som denne rapport kommer til at omhandle.
Dette kunne blandt andet være den basale interaktion med lydkortet der fokuserer på, at lave et program der kan gøre det nemmere for andre programmører.
Dette vil gøre det muligt for dem at kreere geniale løsninger til de mange musikere der er i verden.
Der findes MIDI-biblioteker, der har de løsninger som programmørerne behøver, for at kunne kreere nye programmer.
Disse kan hjælpe brugere der ikke har styr på software programmering, til at lave deres egen musik.
Af denne tankegang udspringer det initierende problem:



\textbf{“Er der områder ved de nuværende muligheder for computergenereret musik der kan forbedres, og er der gamle teknikker der kan bruges i nye sammenhænge?”}



Og derfra kommer problemanalysen, som vil blive udformet i de efterkommende afsnit.