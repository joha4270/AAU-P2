For overhovedet at kunne behandle musik i rapporten, selv om og måske især hvis det er komposition eller redskaber til komposition, er det nødvendigt at beskrive nogle basale musikbegreber
Læren om musik kan danne grundlag for adskillige livslange studier, og nærmest hver eneste kultur og underkultur har sine egne begreber og teknikker til at beskrive musikken


Afsnittets primære fokus er derfor blot en beskrivelse af nogle grundbegreber typisk for det meste musik og navnlig den vestlige


Centralt i alt musik er begrebet  \textit{toner}, lyde der primært består af én mere eller mindre konstant frekvens, hvoraf visse frekvenser opfattes af mennesker som smukkere end andre.

Jo højere frekvensen er, jo lysere opfattes tonen, og jo lavere frekvens jo dybere opfattes denne
\cite{msparkMusic}

\subsubsection{Tolvtoneskalaen}

I moderne musik anvendes tolvtoneskalaen som den standard man bruger til at notere noder, og til at beskrive alle andre skalaer


Skalaen er defineret ud fra en grundtone med frekvens 440hz som navngives A (Frekvensen for A er fundet ved hjælp af eksperimentering, og giver den smukkeste tolvtoneskala)
\cite{msparkMusic}

Tonen med den dobbelte frekvens, 880hz, lyder som en lysere variant af samme tone
Rummet mellem de to toner deles op i 11 nye toner med lige stor indbyrdes afstand
Dette giver en skala af 12 toner, hvor hver 13
tone altid er en lysere variant af den tone man startede fra.

Det samme kan gøres mellem 220hz og 440hz, eller mellem 880hz og 1.760hz osv
\cite{msparkMusic}

Disse toner er traditionelt navngivet \textbf{A, A\#, H, C, C\#, D, D\#, E, F, F\#, G} og \textbf{G\#} 

Læg mærke til at afstanden mellem tonerne er den samme på trods af navngivningen
Denne afstand måles i intervaller, og der siges at være et halvt interval mellem hver tone i tolvtoneskalaen
Hvis man for eksempel arbejder med A\# som grundtone siges E at være tone nummer tre, og F tone tre en halv
\cite{msparkMusic}

\subsubsection{Andre skalaer}

Tolvtoneskalaen har den egenskab, at næsten alle andre toneskalaer kan defineres som en samling af toner fra tolvtoneskalaen
\cite{msparkMusic}

Det gøres traditionelt ved at opskrive en række af intervaller, der angiver afstanden fra den forrige til den næste tone i skalaen.

For eksempel kan man udtrykke den rene durskala som “1 1 \sfrac{1}{2} 1 1 1 \sfrac{1}{2}” (med A som den første tone) svarende til noderne \textbf{A, H, C, D, E, F} og \textbf{G}.

Alle toner tildeles en ny interval-afstand indbyrdes i skalaen, hvorfra det at arbejde inden for en given skala har stor indflydelse på musikkens stil og udtryk
\cite{msparkMusic}

\subsection{Akkorder}

Akkorder er grupper af 3 eller flere toner der spilles samlet, men ikke nødvendigvis samtidig
Dette åbner for mangfoldige og komplekse variationer af simple melodier.

Akkorder defineres med hensyn til intervaller i forhold til grundtonen, som kan være en hvilken som helst tone i skalaen
\cite{msparkMusic}

Ser man eksempelvis på akkorden 1-3-5(durtreklangen) bliver den i den rene durskala C-E-G hvis C er grundtonen, og H-D-F hvis H er grundtonen.

