Computergenereret musik findes i store mængder overalt på internettet, og som tidligere nævnt findes der derudover mange hjælpemidler som musikinteresserede kan bruge til at komponere deres egen musik.
Blandt disse er  Leksikon-sonate, Pymprovisator og  Mario Paint Composer, der gør brug af forud-indspillede lyde.
Blandt de mere professionelle værktøjer kan  Magix Music Maker nævnes.
Dette professionelle stykke software tillader in- og output af MIDI filer, hvilket åbner for interaktionen mellem computer og musikinstrumenter. \cite{CitationNeeded} \todo{Citation Needed}

Til de personer der hellere vil eksperimentere en smule med algoritmer og tilfældighed, kan man bruge et programmeringssprog.
Der findes der flere forskellige biblioteker der kan hjælpe programmøren med at snakke sammen med det integrerede lydkort.


Alternativt kan man lave sit eget format, understøttet af et program der oversætter formatet til noget computerens lydkort kan forstå.
Der er dog visse fordele ved at benytte sig af et standardformat, da et eksisterende filformat kan benyttes i andre stykker software.
Det sikrer også, at filerne stadig kan bruges efter sit eget program ikke længere understøttes.