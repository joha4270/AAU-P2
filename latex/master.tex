%  A simple AAU report template.
%  2015-05-08 v. 1.2.0
%  Copyright 2010-2015 by Jesper Kjær Nielsen <jkn@es.aau.dk>
%
%  This is free software: you can redistribute it and/or modify
%  it under the terms of the GNU General Public License as published by
%  the Free Software Foundation, either version 3 of the License, or
%  (at your option) any later version.
%
%  This is distributed in the hope that it will be useful,
%  but WITHOUT ANY WARRANTY; without even the implied warranty of
%  MERCHANTABILITY or FITNESS FOR A PARTICULAR PURPOSE.  See the
%  GNU General Public License for more details.
%
%  You can find the GNU General Public License at <http://www.gnu.org/licenses/>.
%
\documentclass[10pt,twoside,a4paper,openright]{report}

%Egne Uses:
\usepackage{listings}


\usepackage[Bjornstrup]{fncychap}
% enforce latex compiler to interpret files as utf8 encoded
\usepackage[utf8]{inputenc}
% Make latex understand and use the typographic
% rules of the language used in the document.
\usepackage[danish,english]{babel}
% Use the vector font Latin Modern which is going
% to be the default font in latex in the future.
\usepackage{lmodern}
% Choose the font encoding
\usepackage[T1]{fontenc}

\usepackage{float}
\restylefloat{table}
% load a colour package
\usepackage[table,xcdraw]{xcolor}
%\usepackage{xcolor}
\definecolor{aaublue}{RGB}{33,26,82}% dark blue
% The standard graphics inclusion package
\usepackage{graphicx}
% Set up how figure and table captions are displayed
\usepackage{caption}
\captionsetup{%
  font=footnotesize,% set font size to footnotesize
  labelfont=bf % bold label (e.g., Figure 3.2) font
}
% Make the standard latex tables look so much better
\usepackage{array,booktabs}
% Enable the use of frames around, e.g., theorems
% The framed package is used in the example environment
\usepackage{framed}
%% source code listing.
\usepackage{listings}
% Defines new environments such as equation,
% align and split
\usepackage{amsmath}
% Adds new math symbols
\usepackage{amssymb}
% Use theorems in your document
% The ntheorem package is also used for the example environment
% When using thmmarks, amsmath must be an option as well. Otherwise \eqref doesn't work anymore.
\usepackage[framed,amsmath,thmmarks]{ntheorem}
% Change the headers and footers
\usepackage{fancyhdr}
\setlength{\headheight}{23.0pt} 
% Enable arithmetics with length. Useful when
% typesetting the layout.
\usepackage{calc}

%cross sign
\usepackage{bbding}

% Add the \citep{key} command which display a
% reference as [author, year]
%\usepackage[square]{natbib}
% Appearance of the bibliography
\bibliographystyle{plain}

%%%%%%%%%%%%%%%%%%%%%%%%%%%%%%%%%%%%%%%%%%%%%%%%
% Misc
%%%%%%%%%%%%%%%%%%%%%%%%%%%%%%%%%%%%%%%%%%%%%%%%%
% for multicolumn itemize and enumations
\usepackage{multicol}
% Add bibliography and index to the table of
% contents
\usepackage[nottoc]{tocbibind}
% Add the command \pageref{LastPage} which refers to the
% page number of the last page
% package required in order to use LastPage
\usepackage{lastpage}

\usepackage[
%  disable, %turn off todonotes
  colorinlistoftodos, %enable a coloured square in the list of todos
  textwidth=\marginparwidth, %set the width of the todonotes
  textsize=scriptsize, %size of the text in the todonotes
  ]{todonotes}

% enable english st, nd, rd, th to numbers
  \usepackage{nth}
% Enable hyperlinks and insert info into the pdf
% file. Hypperref should be loaded as one of the
% last packages
\usepackage{hyperref}
\hypersetup{%
	%pdfpagelabels=true,% Results in: "Warning: Option `pdfpagelabels' has already been used,(hyperref) - setting the option has no effect"
	plainpages=false,%
	pdfauthor={Author (s)},%
	pdftitle={Title},%
	pdfsubject={Subject},%
	bookmarksnumbered=true,%
	colorlinks,%
	citecolor=aaublue,%
	filecolor=aaublue,%
	linkcolor=aaublue,% you should probably change this to black before printing
	urlcolor=aaublue,%
	pdfstartview=FitH%
}
\newcommand\JSONnumbervaluestyle{\color{blue}}
\newcommand\JSONstringvaluestyle{\color{red}}

% switch used as state variable
\newif\ifcolonfoundonthisline

\makeatletter

\lstdefinestyle{json}
{
  showstringspaces    = false,
  keywords            = {false,true},
  alsoletter          = 0123456789.,
  morestring          = [s]{"}{"},
  stringstyle         = \ifcolonfoundonthisline\JSONstringvaluestyle\fi,
  MoreSelectCharTable =%
    \lst@DefSaveDef{`:}\colon@json{\processColon@json},
  basicstyle          = \ttfamily,
  keywordstyle        = \ttfamily\bfseries,
}

% flip the switch if a colon is found in Pmode
\newcommand\processColon@json{%
  \colon@json%
  \ifnum\lst@mode=\lst@Pmode%
    \global\colonfoundonthislinetrue%
  \fi
}

\lst@AddToHook{Output}{%
  \ifcolonfoundonthisline%
    \ifnum\lst@mode=\lst@Pmode%
      \def\lst@thestyle{\JSONnumbervaluestyle}%
    \fi
  \fi
  %override by keyword style if a keyword is detected!
  \lsthk@DetectKeywords% 
}

% reset the switch at the end of line
\lst@AddToHook{EOL}%
  {\global\colonfoundonthislinefalse}

\makeatother
% Enable printing of section/chapter name instead of number.
% Requires use of \nameref{<label>} instead of \ref{<label>}.
\usepackage{nameref}

% Dont know what this does, but was required for formatting tables
% as specific way
\usepackage{rotating}
\usepackage{pifont}
\newcommand*\rot{\rotatebox{90}}
\newcommand*\tilt{\rotatebox{70}}
\newcommand*\OK{\ding{51}}

% setup code listings to sane defaults
\lstset{frame=single,
    keepspaces=true,
    keywordstyle=\color{blue},
    numbers=left,
    captionpos=b,
    breakatwhitespace=true,
    breaklines=true,
    tabsize=2,
    basicstyle=\ttfamily\footnotesize}

% used to include eps files
\usepackage{epstopdf}

% This package allows changing of counters in latex
% We use it to use continuous numbering of chapters across parts.
\usepackage{chngcntr}

% Sort of replaces hyperref
\usepackage{bookmark}

% Allows nested figures
\usepackage{subcaption}

\usepackage{pbox}
\usepackage{adjustbox}
\usepackage{wrapfig}

\newtheorem{theorem}{Theorem}

%%todo commands

%example
%the syntax for this command is \todoexample{Some comment or whatever.}
\newcommand{\todoexample}[2][ ]
{
  \ifthenelse{\equal{#1}{inline}}
    {\todo[inline, size=\small, color=blue!40]{\textbf{Example:} #2}}
    {{\includegraphics[height=2ex]{setup/todo/example}} \todo[size=\small, color=blue!40]{\textbf{Example:} #2}}
}


% package inclusion and set up of the document
\input{setup/hyphenations.tex}% 
\input{setup/macros.tex}% my new macros

\begin{document}
%frontmatter
\pagestyle{empty} %disable headers and footers
\pagenumbering{roman} %use roman page numbering in the frontmatter
\pdfbookmark[0]{Front page}{label:frontpage}%
\begin{titlepage}
  \addtolength{\hoffset}{0.5\evensidemargin-0.5\oddsidemargin} %set equal margins on the frontpage - remove this line if you want default margins
  \noindent%
  \begin{tabular}{@{}p{\textwidth}@{}}
    \toprule[2pt]
    \midrule
    \vspace{0.2cm}
    \begin{center}
        \Huge{\textbf{%
      \papername %found in variables.tex
    }}
    \end{center}
    \begin{center}
      \Large{%
          
      }
    \end{center}
    \vspace{0.2cm}\\
    \midrule
    \toprule[2pt]
  \end{tabular}
  \vspace{4 cm}
  \begin{center}
    {\large
      Project Report%Insert document type (e.g., Project Report)
    }\\
    \vspace{0.2cm}
    {\Large
      Group \groupname%found in variables.tex
    }
  \end{center}
  \vfill
  \begin{center}
  Aalborg University\\
  Det Teknisk Naturvidenskabelige Fakultet\\
  Skibbrogade 5\\
  DK 9000 Aalborg
  \end{center}
\end{titlepage}
\clearpage

\thispagestyle{empty}
{\small
\strut\vfill % push the content to the bottom of the page
\noindent Copyright \copyright{} Aalborg University 2016\par
\vspace{0.2cm}
\noindent %Vi kan skrive på denne linje1
}
\clearpage


%\pdfbookmark[0]{English title page}{label:titlepage_en}
\aautitlepage{%
  \englishprojectinfo{%
    \papername%title
  }{%
    \paperthemes%theme
  }{%
    \paperperiod %project period
  }{%
    \groupname % project group
  }{%
    %list of group members
	 \\
	Asger Horn Brorholt \\
	Christian Gundersen Holmgaard \\
	Emil Jensen \\
	Johannes Elgaard \\
  Kenneth Nørgaard Larsen \\
  Mohamad Ibrahim kohistani \\
  Nicklas Højgaard Sneftrup
  }{%
    %list of supervisors
	Kurt Nørmark\\
  }{%
    5 % number of printed copies
  }{%
    \today % date of completion
  }%
}{%department and address
  \textbf{Det Teknisk Naturvidenskabelige Fakultet}\\
  Skibbrogade 5\\
  DK-9000 Aalborg \\
  \href{http://cs.aau.dk}{http://cs.aau.dk}
}{% the abstract
    
} %TODO Ret tittelsider når den endelige tittel lægger fast.
\cleardoublepage
\pdfbookmark[0]{Contents}{label:contents}
\pagestyle{fancy} %enable headers and footers again
\tableofcontents
\listoftodos
%\chapter*{Preface\markboth{Preface}{Preface}}\label{[CHAPTER] Preface}
\addcontentsline{toc}{chapter}{Preface}\vspace{-60pt}

%\section*{Readers Guide}\label{[PREFACE] Readers guide}


\section*{Terminology}\label{[PREFACE] Terminology}

 %TODO Få gruppenavnene ind når vi engang gider.
\cleardoublepage
%mainmatter
\pagenumbering{arabic} %use arabic page numbering in the mainmatter
%%%%%%%%%%%%%%%%%%%%%%%%%%%%%%%%%%%%
%Our Stuff                         %
%%%%%%%%%%%%%%%%%%%%%%%%%%%%%%%%%%%%
\section{Indledning}
I verden findes der mange muligheder for at udleve sine musikalske drømme, i både stor og lille grad.
Det gælder lige fra erfarne musikere der har lavet musik det meste af deres liv, til børn eller andre interesserede der lige er kommet ind i branchen.
Derudover lytter næsten alle mennesker til musik i højere eller mindre grad.
Af netop denne grund er musikområdet et utroligt velbeskrevet, men meget kulturbetonet og mangefacetteret område.


Musik har mange former og kan tage mange retninger.
Det kunne være at den enkelte person kunne lide Jazz, rock eller dansk pop.
Dertil er det også muligt at individet ikke interesserer sig for musikteori.
Herved kan det være at et lille program, som Mario Paint Composer kan tilfredsstille individets musikalske behov.
På samme måde findes der mange kreative måder man kan programmere musik på, og det er dette som denne rapport kommer til at omhandle.
Dette kunne blandt andet være den basale interaktion med lydkortet der fokuserer på, at lave et program der kan gøre det nemmere for andre programmører.
Dette vil gøre det muligt for dem at kreere geniale løsninger til de mange musikere der er i verden.
Der findes MIDI-biblioteker, der har de løsninger som programmørerne behøver, for at kunne kreere nye programmer.
Disse kan hjælpe brugere der ikke har styr på software programmering, til at lave deres egen musik.
Af denne tankegang udspringer det initierende problem:



\textbf{“Er der områder ved de nuværende muligheder for computergenereret musik der kan forbedres, og er der gamle teknikker der kan bruges i nye sammenhænge?”}



Og derfra kommer problemanalysen, som vil blive udformet i de efterkommende afsnit.
\section{Computermusik}
Computergenereret musik findes i store mængder overalt på internettet, og som tidligere nævnt findes der derudover mange hjælpemidler som musikinteresserede kan bruge til at komponere deres egen musik.
Blandt disse er  Leksikon-sonate, Pymprovisator og  Mario Paint Composer, der gør brug af forud-indspillede lyde.
Blandt de mere professionelle værktøjer kan  Magix Music Maker nævnes.
Dette professionelle stykke software tillader in- og output af MIDI filer, hvilket åbner for interaktionen mellem computer og musikinstrumenter. \cite{CitationNeeded} \todo{Citation Needed}

Til de personer der hellere vil eksperimentere en smule med algoritmer og tilfældighed, kan man bruge et programmeringssprog.
Der findes der flere forskellige biblioteker der kan hjælpe programmøren med at snakke sammen med det integrerede lydkort.


Alternativt kan man lave sit eget format, understøttet af et program der oversætter formatet til noget computerens lydkort kan forstå.
Der er dog visse fordele ved at benytte sig af et standardformat, da et eksisterende filformat kan benyttes i andre stykker software.
Det sikrer også, at filerne stadig kan bruges efter sit eget program ikke længere understøttes.
\section{Musik Teori}
For overhovedet at kunne behandle musik i rapporten, selv om og måske især hvis det er komposition eller redskaber til komposition, er det nødvendigt at beskrive nogle basale musikbegreber
Læren om musik kan danne grundlag for adskillige livslange studier, og nærmest hver eneste kultur og underkultur har sine egne begreber og teknikker til at beskrive musikken


Afsnittets primære fokus er derfor blot en beskrivelse af nogle grundbegreber typisk for det meste musik og navnlig den vestlige


Centralt i alt musik er begrebet  \textit{toner}, lyde der primært består af én mere eller mindre konstant frekvens, hvoraf visse frekvenser opfattes af mennesker som smukkere end andre.

Jo højere frekvensen er, jo lysere opfattes tonen, og jo lavere frekvens jo dybere opfattes denne
\cite{msparkMusic}

\subsubsection{Tolvtoneskalaen}

I moderne musik anvendes tolvtoneskalaen som den standard man bruger til at notere noder, og til at beskrive alle andre skalaer


Skalaen er defineret ud fra en grundtone med frekvens 440hz som navngives A (Frekvensen for A er fundet ved hjælp af eksperimentering, og giver den smukkeste tolvtoneskala)
\cite{msparkMusic}

Tonen med den dobbelte frekvens, 880hz, lyder som en lysere variant af samme tone
Rummet mellem de to toner deles op i 11 nye toner med lige stor indbyrdes afstand
Dette giver en skala af 12 toner, hvor hver 13
tone altid er en lysere variant af den tone man startede fra.

Det samme kan gøres mellem 220hz og 440hz, eller mellem 880hz og 1.760hz osv
\cite{msparkMusic}

Disse toner er traditionelt navngivet \textbf{A, A\#, H, C, C\#, D, D\#, E, F, F\#, G} og \textbf{G\#} 

Læg mærke til at afstanden mellem tonerne er den samme på trods af navngivningen
Denne afstand måles i intervaller, og der siges at være et halvt interval mellem hver tone i tolvtoneskalaen
Hvis man for eksempel arbejder med A\# som grundtone siges E at være tone nummer tre, og F tone tre en halv
\cite{msparkMusic}

\subsubsection{Andre skalaer}

Tolvtoneskalaen har den egenskab, at næsten alle andre toneskalaer kan defineres som en samling af toner fra tolvtoneskalaen
\cite{msparkMusic}

Det gøres traditionelt ved at opskrive en række af intervaller, der angiver afstanden fra den forrige til den næste tone i skalaen.

For eksempel kan man udtrykke den rene durskala som “1 1 \sfrac{1}{2} 1 1 1 \sfrac{1}{2}” (med A som den første tone) svarende til noderne \textbf{A, H, C, D, E, F} og \textbf{G}.

Alle toner tildeles en ny interval-afstand indbyrdes i skalaen, hvorfra det at arbejde inden for en given skala har stor indflydelse på musikkens stil og udtryk
\cite{msparkMusic}

\subsection{Akkorder}

Akkorder er grupper af 3 eller flere toner der spilles samlet, men ikke nødvendigvis samtidig
Dette åbner for mangfoldige og komplekse variationer af simple melodier.

Akkorder defineres med hensyn til intervaller i forhold til grundtonen, som kan være en hvilken som helst tone i skalaen
\cite{msparkMusic}

Ser man eksempelvis på akkorden 1-3-5(durtreklangen) bliver den i den rene durskala C-E-G hvis C er grundtonen, og H-D-F hvis H er grundtonen.


\section{Redskaber til Projektet C\#}
For bedst at opfylde projektets læringsmål arbejdes i det objektorienterede programmeringssprog C\#.
Sproget har et indbygget bibliotek, \emph{System.Media}, til at håndtere bølgebaserede lydfiler. Biblioteket kan afspille musik fra en fil, samt skrive lydbølger til en wavefile. Denne repræsentation af lyd er fortræffelig til at lagre i forvejen optagede lydklip, men det at generere musik fra grunden bliver enten ekstremt teknisk eller meget begrænset.\todo{Citation Needed} \cite{CitationNeeded}

Skulle musik genereres med kode, er det mere oplagt blot at arbejde i toner og instrumenter, hvilket understøttes af MIDI-standarden.  Fordelen ved MIDI-formatet er at det er meget simpelt og filerne fylder generelt mindre end bølge-baserede lydformater. MIDI formatet er dog ikke lige så fleksibelt som de bølgebaserede formater, bl.a. da det ikke er baseret på bølger og at man derfor ikke kan lave lyrik i MIDI filer. Til denne rapports formål er det dog ikke et problem, da lyrik ligger uden for rapportens område, men blot kan lægges over i en evt. senere gennemarbejdning af musikken, skulle nogen ønske det.
\todo{Skriv: "Vi vælger at bruge MIDI"}
\section{Om MIDI formatet}
Under musikkens digitale revolution var der behov for et standardiseret signal til kommunikation mellem forskellige elektronsike komponenter, samt behov for at spille på flere instrumenter synkront, og for at spille det samme gentagne gange.
\cite{Citation Needed}\todo{Citation Needed}.
Dertil designede man \emph{Musical Instrument Digital Interface}, MIDI-formatet.
 

 

Udover at være et format hvormed elektroniske instrumenter kommunikerer, kan MIDI-meddelelserne bruges som et filformat der kan gemmes, redigeres og afspilles på en computer\cite{Citation Needed}\todo{Citation Needed}.



\begin{center}
\begin{tabular}{ | l | l | }
	\hline
	\multicolumn{2}{|c|}{Fordele}\\ \hline
	MIDI 		& Bølgeform\\ \hline \hline
	Kompakt 	& Detaljeret\\ \hline
	Simpelt 	& Fleksibelt\\ \hline
\end{tabular}
\end{center}
\paragraph{Kompakt vs.
Detaljeret}

I modsætning til bølgeformaterne definerer MIDI ikke lyd direkte.
Det er en 

abstraktion over den rigtige lyd, og understøtter kun nogle få muligheder i forhold til hvad der er muligt for et bølgeformat.


Dette har diverse fordele og ulemper, for eksempel understøtter MIDI ikke ting som sang, specielle effekter, mere end 16 instrumenter ad gangen eller brug af obskure instrumenter.
Til gengæld er abstraktionen meget lettere at arbejde med når man skal skabe musik fra bunden, hvis man ikke ønsker at beskæftige sig direkte med tonegenerering og i stedet ønsker at arbejde med mere sofistikerede instrumenter.


\subsection {Eksperimentering med MIDI-formatet}

I starten var idéen at afteste forskellige former for autogenerering ved hjælp af MIDI formatet, hvor der skulle tages udgangspunkt i den musikteori der tidligere var blevet belyst.
Projektets problemfelt havde i starten retning mod autogenereret computermusik, og her blev de mulige metoder der kunne anvendes undersøgt.
    

Ud fra dette blev der udarbejdet nogle forsøgsprogrammer i C\# med hjælp fra et bibliotek kaldet midi-dot-net.

I alt blev der dannet fire programmer ud fra eksperimenteringen med MIDI og autogenerering af musik.





\subsubsection{Musik generering ved tilfældighed}

De første overvejelser med musik-programmering, var at teste hvorvidt der kunne programeres et musikstykke i C\#.
Udfordringen var derfor først, at teste midi-dot-net’s muligheder for at få lyd igennem lydkortet og ud til højtalerne.
Derefter var målet at generere et musikstykke med toner valgt helt tilfældigt..
Det der skulle angives på forhånd, var de toner computeren skulle generere musikken ud fra, da der var sat en default-værdi for tempoet i musikkemn.
 

De tilfældigt valgte toner lød lige så dårligt i sekvens som man kunne forvente, men demonstrerede fint mulighederne for at generere musik med midi-dot-net.
Derudover virkede den fuldstændigt tilfældige musik som en kontrolgruppe som andre genereringsteknikker kunne sammenlignes med.
Nu kunne programmet testes, forsøgene bevægede sig over mod at implementere akkorder og rytmer i musikstykket.
Det blev klart, at dette var dårligt implementeret i midi-dot-net.
\cite{CitationNeeded}\todo{Citation Needed}



\subsubsection{Semi-tilfældig musik med Kvintcirklen}

Efter det første program var færdiggjort og der kunne skabes tilfældige toner, gik næste forsøg i gang.


Kvintcirklen placerer tonerne fra tolvtoneskalaen i en cirkel og angiver for hver tone, hvilke andre toner der lyder godt når de spilles efter denne.
Programmet fungerede ved at hver tone fra kvintcirklen blev stillet op i et array, og startede ved en tilfældig tone og derefter gik videre til en tilfældig tone blandt de mulige ifølge kvintcirklen.
Programmet skabte ikke nogen stor musisk kunst , men det dannede musik der lød bedre en fuldstændig tilfældige toner.
\cite{citation+needed}



\subsubsection{Musik generering ved Cellular Automata}

Cellular automata er en type simulationer hvor “celler”(oftest repræsenteret med firkanter) ændrer stadie baseret på cellerne omkring dem.
Den simpleste form, elementær cellular automaton, består af en endimensionel stribe celler der alle kan have et af to forskellige stadier, og hvis fremtidige stadie bestemmes ud fra cellen selv og dens to naboceller.
Denne meget simple præmis giver 256 mulige regelsæt for, hvornår celler opstår eller forsvinder.\cite{wolframCA}

Når dette er brugt til at lave musik ud fra, ser man “cellerne” ude i prompten som kører inde i programmet, imens det spiller toner alt efter hvordan cellerne begår sig.
Dette betyder, at der bliver lavet musik ud fra en tilfældig, men også forudsigelig metode, hvorpå det faktisk kan komme til at lyde okay, alt efter hvordan tonerne er.
\cite{CitationNeeded}\todo{Citation Needed}



\subsubsection{Musik generering via.
genetiske algoritmer}

Det sidste af de programmer der blev lavet bruger en genetisk algoritme, denne algoritme er en rekursions algoritme, som kører efter noget lign.
evolutionsteorien, dette betyder at den skal bruge nogle krav og nuværende objekter.
Algoritmen tager en række af objekter som parametre.
Samt et objekt med de krav der stilles til objekterne.
Hvert objekt består af en række værdier der hver repræsenterer en tone.


Når algoritmen kører vil den da prøve at kombinere værdier fra objekterne, så den kan leve op til de krav som er stillet.
Hvis en af kombinationerne mellem objekterne stemmer overens med kravene, afsluttes algoritmen, ellers vil algoritmen først blive kaldt i slutningen af kombinationerne, hvor den så bare tager resultatet af kombinationerne som de nye objekter.

Algoritmen blev kørt det antal gange den fik som parametre hvilket som standard var 100 iterationer, derefter blev alle de overlevende objekter omskrevet til en liste af toner som kunne afspilles inden næste kald af algoritmen.
Ud fra dette forsøg kunne der konkluderes at man ikke kunne generere ny musik på denne måde, hvis man derimod anvendte teorien anderledes kunne man generere musik ud fra en lang række parametre.
Disse parametre kunne da bestemmes på compile-time af brugeren, så musikken konstant ændrede sig til hvad brugeren ville.
Dette kunne være et interessant emne at arbejde med.\cite{CitationNeeded}\todo{Citation Needed}

\section{Eksisterende C\#-biblioteker der følger MIDI-standarden} \todo{renskriv}
\todo{Meta-tekst her pls}

Efter eksperimenteringerne stod det klart at det C\# bibliotek der var blevet anvendt i forsøgene havde en del mangler,  og ikke ville være effektivt at benytte i et professionelt program. 
Derfor er der blevet fundet og undersøgt en række C\# biblioteker der kan anvendes til arbejde i MIDI-formatet.



Tabel over eksisterende C\#- biblioteker med deres fordele og ulemper:

\begin{center}
	\todo{Move links to somewhere}
	\resizebox{12cm}{!}{
		\begin{tabular}{| r || p{4cm} | p{4cm}  | p{4cm}  | p{4cm}  | p{4cm}  | }
			\hline 
			Navn 	& MS API 				& C\# MIDI Toolkit		& naudio									& midi\-dot\-net			& MIDI.NET \\ \hline 
			%Link 	& \url{https://msdn.microsoft.com/en-us/library/dd798495(v=vs.85).aspx} &  \url{http://www.codeproject.com/Articles/6228/C-MIDI-Toolkit} &  \url{https://naudio.codeplex.com/} &  \url{https://code.google.com/archive/p/midi-dot-net/}  \url{https://github.com/jstnryan/midi-dot-net} &  \url{https://midinet.codeplex.com/} \\ \hline
			Fordele	& God dokumentation.	& Some documentation 	& Filer og enheder som både input og output	& Intuitiv og let at bruge 	& Filer og enheder som både input og output \\ 
					& Tæt på metallet, ingen overhead 	& Filer og enheder som både input og output & Kan meget & God interface til timing 	& \\ \hline
			Ulemper	& Tæt på metallet, programmøren skal alt & Interface er ineffektivt og svært at forstå &  \(stor\) & Ingen filer & Ikke eksisterende dokumentation \\ 
					& P/Invoke for at kommunikere med C\#/.NET & dokumentationen er ikke så godt skrevet &  & Har ikke hele MIDI standarden & Forvirrende og ineffektiv interface \\
					& & &  & dårlig/ingen dokumentation & Indre detaljer lækker til brugeren \\  \hline
		\end{tabular}
	}
\end{center}


\subsection{MS API}

Microsoft MIDI API er et godt dokumenteret API som forklarer alle det funktioner og samtidig kommer med eksempler på hvordan det kunne bruges og forklare alle parameter der bruges.

API’en er tæt på MIDI-formatet hvilket giver brugeren fuld kontrol over hvordan lyden skal genereres men kræver tilsvarende mere arbejde som andre biblioteker klare for brugeren. 
Den er implementeret i nativ kode så den skal bruge P/Invoke for at kommunikere med C\#.NET.



\subsection{C\# MIDI Toolkit}

http://www.codeproject.com/Articles/6228/C-MIDI-Toolkit

Tydeligvis meget populært. 
Lavet af en hobbyist som stadig vedligeholder det under MIT licens(hvad det end er).

Så vidt det ses, så forsøger biblioteket at være være meget fleksibelt og objektorienteret.  
MIDI-input kommer ind i den ene ende og ud af den anden, hvor det er muligt for brugeren af biblioteket at smide et komponent ind i midten der modificerer MIDI-inputtet. 
Der ses ikke at der decideret bliver anvendt en MIDI-fil, men det er lavet til at læse og afspille fra hvad som helst.

Hertil findes der heller ikke nogen dokumentation, og artiklens forsøg på at forklare hvad der er op og ned er håbløs. 
Ud over det ser det professionelt ud og der er i hvert fald nogen der bruger det.

\subsection{NAudio}

https://naudio.codeplex.com/

Naudio er et meget stort bibliotek til at arbejde med al slags lyd herunder afspilning og modificering af diverse lydfilformater som wav, mp3 og MIDI.

MIDI er altså en lille del af dette biblioteks formåen, men den har alle de funktioner som MIDI har brug for som indlæsning af filer,output af lyd og konvertering af MIDI-filer til andre formater. 
Biblioteket  gør brug af enums med hensyn til når der skal oprettes events kan man skrive navnet på hvilken type event man vil have oprettet i stedet for at skrive et binært tal til at oprette eventen, hvilket er meget brugervenligt.

\subsection{midi-dot-net}

https://code.google.com/archive/p/midi-dot-net/

midi-dot-net er et C\# bibliotek som giver adgang til Microsofts MIDI API i C\# i en objektorienteret abstraktion. 
Den ligger sig meget tæt op af Microsofts MIDI API hvor de eneste abstraktioner er at relaterede metoder er flyttet til objekter og det er ikke muligt at sende rå bytes til MIDI API’en som er bygget op af noder med pitch og velocity. 
Derudover er der er der en række klasser som skal gøre det lettere at komme i gang med at producere musik. Der er en timer klasse der kan håndtere at afspille MIDI events på de rigtige tidspunkter og at automatisk at stoppe med at afspille dem igen. Derudover har den en akkord klasse som tillader brugeren at afspille hele akkorder.

\subsection{MIDI.NET}

https://midinet.codeplex.com/ 

MIDI.NET tillader alle .NET udviklere til at få adgang til MIDI uden at udvikleren behøver at gøre brug af P/Invokes.

Biblioteket læner sig meget over til MIDI formatet den abstrahere ikke så meget.

Efter en undersøgelse af MIDI.NET bibloteket er der blevet fundet ud af at MIDI.NET bibloteket ikke har særlig meget dokumentation udenfor selve koden det dokumentationen kommer meget kort omkring MIDI ports og MIDI beskeder. 
Den dokumentation som er i selve koden giver en kort beskrivelse af de forskellige af metoder og variabler dog er der ingen overordnet forklaring omkring selve bibliotekets design, brug og funktionaliteter.

Udfra hvad man kan se i koden ser det ud til at biblioteket er istand til at tage MIDI filer som input og som output kan den udskrive en MIDI-fil eller afspille via microsofts interface.

Der er blevet taget nogle ikke så gode valg da det kom til implementeringen af classen MidiBufferStream da der programmøren har valgt at bruge unsafe pointer aritmetik da det helt klart ville være at foretrække at det blev løst på en anden måde som ikke er unsafe.  

\section{Opsummering}
Musikteori har givet projektet den nødvendige viden om musik og hvordan det udføres hos musikerne.
Med denne viden blev det klart at teorien skulle forenkles så de interesserede programmører kan gøre brug af det udarbejdede bibliotek.
Computermusik kom derefter ind på hvilke konkurrenter projektet har på det nuværende marked, og gav en antydning om hvilke muligheder der er på området.
MIDI delen er derimod  

I rapporten er der blevet gennemgået musikteori, computermusik, og MIDI implementation.
Rapporten går derfor nu videre til at beskrive hvad og hvordan det endelige produkt er blevet udformet.
 
\section{Ønskværdige egenskaber til et bibliotek}
\section{overvejelser over features}
\section{Problemformulering}
\section{Løsningsforslag}

\printbibliography[heading=bibintoc]
\label{bib:mybiblio}
\appendix
%Appendix goes here.

\end{document}
